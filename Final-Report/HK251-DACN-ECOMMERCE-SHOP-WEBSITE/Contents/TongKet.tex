\section{Tổng kết}

\subsection{Nhận xét}

Thông qua giai đoạn 1 - đồ án chuyên ngành, nhóm đã tiến hành nghiên cứu và phân tích các ưu nhược điểm của các hệ thống bán hàng của các cửa hàng thời trang lớn hiện đang có trên thị trường. Từ đó có được cái nhìn tổng quan về nhu cầu cấp thiết cần có một hệ thống bán hàng thời trang cho các cửa hàng thời trang vừa và nhỏ. Hệ thống đó cần đáp ứng sở thích mua sắm tiện lợi của người khách hàng vừa phải đáp ứng nhu cầu quản lý cửa hàng của những người chủ cửa hàng vừa và nhỏ. \\

Tiếp theo đó, nhóm đã tiến hành phân tích nhu cầu của người dùng hệ thống theo từng vai trò và chức năng họ có thể thực hiện được bên trong hệ thống. Trong đó các tính năng nổi bật có thể kể đến đó là thực hiện mua ngay sản phẩm, mua sản phẩm từ giỏ hàng, mua lại sản phẩm từ đơn hàng trước đó. Không những vậy, người khách hàng còn có thể so sánh thông tin các sản phẩm trước khi mua hàng, có thể thanh toán đơn hàng thông qua cổng thanh toán bên thứ 3 nhanh chóng, tiện lợi. Để tăng cao trải nghiệm sử dụng dịch vụ bên cửa hàng, hệ thống còn hỗ trợ việc đề xuất các sản phẩm hot trend, theo lịch sử mua hàng, duyệt web, hỗ trợ việc gợi ý các sản phẩm trong cùng một bộ (buy full set), gợi ý kích cỡ phù hợp của sản phẩm đối với người khách hàng, nhắn tin realtime để người khách hàng có thể nhận sự trợ giúp từ phía cửa hàng khi mua và sử dụng sản phẩm. Ngoài ra, hệ thống còn cho phép người khách hàng thực hiện hủy đơn hàng hoặc yêu cầu trả hàng hoàn tiền khi đơn hàng nhận được không đúng mô tả sản phẩm cửa hàng đã đưa ra. Còn đối với người quản trị viên cửa hàng, họ có thể quản lý danh mục sản phẩm, sản phẩm, voucher, đơn hàng, thông tin nhân viên của cửa hàng. Ngoài ra người quản trị viên có thể thống kê và biết được tình hình kinh doanh cửa của hàng thông qua các biểu đồ về doanh thu, tổng lượng đơn hàng bán, tổng số đơn hàng giao thành công hoặc đã hủy,... theo tuần / tháng / năm. Hệ thống còn giúp người quản trị viên biết được sản phẩm nào bán chạy nhất, nhân viên nào có doanh số cao nhất,..\\

Bên cạnh đó, trong đồ án chuyên ngành, nhóm đã tiến hàng thiết kế được kiến trúc bậc cao của hệ thống theo kiến trúc Three Tier, thiết kết cơ sở dữ liệu, thiết kế sơ đồ hoạt động, sơ đồ tuần tự, sơ đồ lớp. Đây là những tài liệu quan trọng sẽ hỗ trợ rất lớn cho việc phát triển, vận hành, bảo trì và nâng cấp hệ thống sau này. \\

Hệ thống bán hàng trực tuyến đã được xây dựng theo kiến trúc three-tier với nền tảng backend NestJS, Prisma ORM kết nối PostgreSQL, xác thực người dùng bằng JWT và Google OAuth, lưu trữ media trên AWS S3, và tích hợp thanh toán VNPay. Kiến trúc module hoá ở phía backend (sản phẩm, đơn hàng, vận chuyển, khuyến mãi, người dùng, truyền thông) giúp tách bạch trách nhiệm, dễ mở rộng và bảo trì. Bộ API tuân thủ chuẩn REST và có tài liệu hoá OpenAPI, thuận tiện cho phát triển frontend và kiểm thử. \\

Về mặt vận hành, các luồng nghiệp vụ cốt lõi (duyệt/tìm kiếm sản phẩm, giỏ hàng, đặt hàng, xử lý đơn hàng, trả hàng/hoàn tiền, thông báo, chat hỗ trợ) đã được thiết kế giao diện chi tiết cho cả phía Khách hàng, Admin và Nhân viên, đảm bảo tính nhất quán trải nghiệm. Các giao diện này sẽ giúp người chủ cửa hàng sẽ có cái nhìn trực diện và hình dung được các tính năng mà hệ thống sẽ có. Các tài liệu này sẽ giúp đội ngũ phát triển hiện thực đúng chức năng và các yêu cầu mà người chủ cửa hàng đưa ra. \\

\newpage
\subsection{Hướng phát triển đồ án tốt nghiệp}

\begingroup
\renewcommand{\arraystretch}{1.4} % Tăng chiều cao dòng
\centering
\begin{xltabular}{\textwidth}{|c|c| >{\hsize=0.8\hsize}X | >{\hsize=1.2\hsize}X |}
	% --- PHẦN 1: TIÊU ĐỀ CHO TRANG ĐẦU TIÊN ---
	\hline
	\multicolumn{4}{|c|}{\bfseries \textcolor{red}{ĐỀ TÀI: PHÁT TRIỂN HỆ THỐNG BÁN HÀNG TRỰC TUYẾN CHO CỬA HÀNG THỜI TRANG}} \\
	\hline
	\multicolumn{4}{|c|}{\bfseries \textcolor{red}{LỊCH TRÌNH ĐỒ ÁN TỐT NGHIỆP: TỪ 12/01/2026 ĐẾN HẾT 27/04/2026}} \\
	\hline
	\rowcolor{excelblue}
	\multicolumn{1}{|c|}{\textbf{Tuần}} & 
	\multicolumn{1}{|c|}{\textbf{Ngày}} & 
	\multicolumn{1}{|c|}{\textbf{Công việc}} & 
	\multicolumn{1}{|c|}{\textbf{Chi tiết nội dung công việc}} \\
	\hline
	\endfirsthead
	% --- PHẦN 2: TIÊU ĐỀ LẶP LẠI KHI SANG TRANG MỚI ---
	\hline
	\rowcolor{excelblue}
	\multicolumn{1}{|c|}{\textbf{Tuần}} & 
	\multicolumn{1}{|c|}{\textbf{Ngày}} & 
	\multicolumn{1}{|c|}{\textbf{Công việc}} & 
	\multicolumn{1}{|c|}{\textbf{Chi tiết nội dung công việc}} \\
	\hline
	\endhead
	% --- PHẦN 3: CHÂN TRANG (Mỗi khi hết trang) ---
	\hline
	\multicolumn{4}{r}{\textit{Còn tiếp sang trang sau...}} \\
	\endfoot
	% --- PHẦN 4: CHÂN TRANG (Kết thúc bảng) ---
	\endlastfoot
	% --- NỘI DUNG DỮ LIỆU ---
	% Tuần 1
	1 & 12/01/2026 & 
		- Nâng cấp UI cải thiện trải nghiệm người dùng. \newline
		- Thực hiện việc đề xuất và cá nhân hóa cho từng người dùng. \newline
		- Viết báo cáo hàng tuần. \newline
	 & 
		\begin{itemize}[leftmargin=*, nosep, after=\vspace{-\baselineskip}]
			\item Hiện thực bộ lọc: Xây dựng api hỗ trợ lọc đa điều kiện. (Lọc theo khoảng giá, màu sắc, kích thước, danh mục sản phẩm, hot trend,..).
			\item Tối ưu Query để xử lý việc truy xuất trên nhiều bảng trong cơ sở dữ liệu để lấy được kết quả theo bộ lọc.
			\item Hiện thực logic đề xuất theo danh mục (đề xuất bằng các sản phẩm đang bán chạy trong danh mục.).
			\item Hiện thực logic đề xuất theo lịch xử duyệt web và mua hàng của người dùng (đề xuất giải thuật k-Nearest Neighbors).
			\item Hiện thực logic hỗ trợ việc gợi ý các sản phẩm trong cùng một bộ (buy full set), gợi ý kích cỡ phù hợp của sản phẩm đối với người khách hàng.
			\item Hiện thực UI tương ứng với các tính năng trên.
			\item Viết báo cáo hàng tuần
		\end{itemize}
	
	\\ 
	\cline{1-2}
	% Tuần 2 (Copy nhiều dòng để test sang trang)
	2 & 19/01/2026 &&\\ \hline
	3 & 26/01/2026 & 
		-Tích hợp hệ thống thanh toán bên thứ 3: VNPay, MOMO. \newline
		- Viết báo cáo hàng tuần .\newline
	 & 
		\begin{itemize}[leftmargin=*, nosep, after=\vspace{-\baselineskip}]
			\item Tiến hành xây dựng api tích hợp cho việc kết nối cổng thanh toán vnpay. Momo,
			\item Hiện thực các UI tương ứng nếu có.
			\item Xử lý việc cập nhật trạng thái thanh toán của đơn hàng sau khi thanh toán thành công. 
			\item Hiện thực thông báo realtime khi thanh toán thành công / thất bại.
			\item Viết báo cáo hàng tuần
		\end{itemize}
	\\ 
	\cline{1-2}
	4 & 02/02/2026 &&\\ \hline
	5 & 09/02/2026 &
	- Tích hợp hệ thống thông báo realtime cho các sự kiện hệ thống. \newline
	- Xây dựng tính năng nhắn tin realtime. \newline
	- Viết báo cáo hàng tuần. \newline
	&
	\begin{itemize}[leftmargin=*, nosep, after=\vspace{-\baselineskip}]
		\item Xây dựng tính năng thông báo và nhắn tin realtime dựa trên thư viện Socket.io của Nestjs hỗ trợ.
		\item Viết báo cáo hàng tuần
	\end{itemize}
	\\ 
	\cline{1-2}
	6 & 16/02/2026 &&\\ \hline
	7 & 23/02/2026 &
	- Tích hợp các chức năng báo cáo trên dashboard của người quản trị viên. \newline
	- Viết báo cáo hàng tuần. \newline
	&
	\begin{itemize}[leftmargin=*, nosep, after=\vspace{-\baselineskip}]
		\item Xây dựng các apis query kết quả tổng kết doanh thu, tổng kết đơn hàng, tổng kết số lượng đơn giao thành công, đơn trả hàng,….
		\item Query các thông số sản phẩm bán chạy nhất, nhân viên doanh số cao nhất, danh sách số lượng hàng tồn còn lại trong kho,…
		\item Hiện thực và tinh chỉnh UI theo figma đã có.
		\item Viết báo cáo hàng tuần
	\end{itemize}
	\\ 
	\cline{1-2}
	8 & 02/03/2026 &&\\ \hline
	9 & 09/03/2026 &
	- Hiện thực tinh chỉnh giao diện dark mode. \newline
	- Hiện thực tích hợp đa ngôn ngữ: Tiếng Việt / Tiếng Anh. \newline
	- Viết báo cáo hàng tuần. \newline
	&
	\begin{itemize}[leftmargin=*, nosep, after=\vspace{-\baselineskip}]
		\item Tích hợp thư viện react-i18next cho việc cấu hình ngôn ngữ việt nam, english.
		\item Viết báo cáo hàng tuần
	\end{itemize}	
	\\ 
	\cline{1-2}
	10 & 16/03/2026 &&\\ \hline
	11 & 23/03/2026 &
	- Thực hiện testing và đánh giá:
	Unit testing,  integration testing, system testing và acceptance testing. \newline
	- Viết báo cáo hằng tuần. \newline
	&
	\begin{itemize}[leftmargin=*, nosep, after=\vspace{-\baselineskip}]
		\item Thực hiện kiểm thử tự động bằng Selenium.
		\item Kiểm tra và cập nhật validate và transform dữ liệu ở front-end và cả backend. Dùng các thư viện được hỗ trợ sẵn như class-validate, class-tranformer.
		\item Kiểm tra và cập nhật lỗ hổng để lộ thông tin người dùng ví dụ như mật khẩu người dùng.
		\item Kiểm tra và cập nhật vấn đề N + 1 query .
		\item Kiểm tra và cập nhật index cho các bảng thường xuyên được truy xuất ví dụ Order, Payment, Product, ProductVariant.
		\item Kiểm tra và cập nhật phân trang trong các api trả về kết quả là một danh sách các đối tượng.
		\item Kiểm tra và cập nhật transaction cho các api có nghiệp vụ xử lý phức tạp trên nhiều bảng một lúc. Đảm bảo rollback ngay lập tức khi có lỗi. 
		\item Thực hiện try catch cho việc bắt lỗi hệ thống.
		\item Check và cập nhật lỗi IDOR.
		\item Kiểm tra và cập nhật để tránh lỗi SQL Injection khi dùng các Raw Query.
		\item Kiểm tra và cập nhật các lỗi để tránh tấn công Cross-Site Scripting.
		\item Viết báo cáo hàng tuần
		
	\end{itemize}
	\\ 
	\cline{1-2}
	12 & 30/03/2026 &&\\ 
	\cline{1-2}
	13 & 06/04/2026 &&\\ 
	\hline
	14 & 13/04/2026 &
	-Deploy và demo. \newline
	- Hoàn thiện báo cáo đồ án tốt nghiệp. \newline
	&
	\begin{itemize}[leftmargin=*, nosep, after=\vspace{-\baselineskip}]
		\item Triển khai hệ thống lên server.
		\item Cấu hình timeout cho server.
		\item Cấu hình iptable để hạn chế số lượng request mà 1IP có thể gửi trong 1s, chặn IP khi một IP nào đó khi nó gửi quá số lượng request cho phép để hạn chế ảnh hưởng của Dos/DDoS 
		\item Cấu hình CDN Cloudflare để hạn chế ảnh hưởng của Dos/DDoS.
		\item Cấu hình load balancing bằng Nginx.
		\item Viết báo cáo hàng tuần.
		\item Quay chụp video, hình ảnh demo dự án.
	\end{itemize}
	\\ 
	\cline{1-2}
	15 & 20/04/2026 &&\\ \hline
	\caption{Lịch trình chi tiết thực hiện đồ án tốt nghiệp vào học kỳ 252.} \label{tab:lich_trinh_do_an_tot_nghiep_hk252}
\end{xltabular}
\endgroup

\newpage