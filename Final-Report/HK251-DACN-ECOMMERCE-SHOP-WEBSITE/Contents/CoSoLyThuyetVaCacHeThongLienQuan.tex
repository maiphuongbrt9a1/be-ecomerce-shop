\section{Phân tích tổng quan}

\subsection{Các hệ thống có liên quan trên thị trường}

\subsubsection{H \& M}
\subsubsubsection{Giới thiệu H \& M}
H \& M là ví dụ điển hình của một hệ thống bán lẻ quy mô lớn (Large-scale Retailer). Xuất phát từ Thụy Điển, thương hiệu này đã xây dựng một chuỗi cung ứng khổng lồ kết nối hàng trăm nhà máy sản xuất với hệ thống kho vận (Logistics) hiện đại để phục vụ cả kênh bán hàng truyền thống (Offline Stores) và thương mại điện tử (E-commerce). Điểm mạnh của H&M nằm ở khả năng cập nhật thiết kế nhanh chóng (Lead time ngắn) và quản lý tồn kho hiệu quả dựa trên dữ liệu lớn, giúp thương hiệu này luôn giữ vững vị thế dẫn đầu trong phân khúc thời trang bình dân toàn cầu.\cite{H \& M}

\begin{figure}[H]
	\centering
	\includegraphics[width=0.8\textwidth]{Images/Brand_Imgs/HandM.png}
	\vspace{0.5cm}
	\caption{Trang chủ H \& M \cite{H \& M}}
	\label{fig: Trang chủ H \& M}
\end{figure}

\subsubsubsection{Đặc điểm kỹ thuật nổi bật}

Hệ thống web/app H\&M tại Việt Nam triển khai các chức năng kỹ thuật chính như:

\begin{itemize}
  \item \textbf{Hệ thống thành viên số hóa (H\&M Member):} đăng ký, tích điểm, voucher, quản lý giao dịch qua tài khoản số.
  \item \textbf{Tích hợp Omni-channel / kiểm tra tồn kho tại cửa hàng (Find in-store):} cho phép người dùng kiểm tra tình trạng hàng tại cửa hàng gần họ từ web/app.
  \item \textbf{Danh mục \& bộ lọc nâng cao:} sản phẩm được phân loại chi tiết (giới tính, độ tuổi, dòng Conscious, thể thao...), kèm bộ lọc kích cỡ, màu sắc, nhãn bền vững.
  \item \textbf{Thanh toán \& đổi trả trực tuyến:} hỗ trợ nhiều phương thức thanh toán (thẻ, COD tuỳ thị trường) và chế độ đổi/trả trong 30 ngày qua hệ thống web/app.
  \item \textbf{Nỗ lực ứng dụng AR/virtual try-on:} thử nghiệm công nghệ ảo nhưng chưa áp dụng rộng rãi cho tất cả sản phẩm.
\end{itemize}

\subsubsubsection{Đánh giá hệ thống}

\textbf{Ưu điểm nổi bật:}
\begin{itemize}
  \item Liên kết mượt giữa kênh online và cửa hàng vật lý (omni-channel).
  \item Quy trình đổi trả \& CRM thành viên được số hóa, thuận tiện cho khách hàng.
  \item Hỗ trợ lọc sản phẩm theo yếu tố bền vững (``Conscious''), đáp ứng xu hướng xanh.
  \item Hạ tầng hỗ trợ các chiến dịch ra mắt nhanh (fast fashion) và drop giới hạn.
\end{itemize}

\textbf{Nhược điểm \& điểm cần cân nhắc:}
\begin{itemize}
  \item Tính năng AR/try-on chưa được triển khai đồng nhất cho mọi sản phẩm.
  \item Đa dạng size/phom dáng có thể gây nhầm lẫn, cần dữ liệu kích thước chi tiết hơn hoặc công cụ so sánh size.
  \item Hệ thống gợi ý cá nhân hoá còn hạn chế, chưa thực sự tinh chỉnh theo người dùng.
  \item Nguy cơ quá tải hạ tầng vào các đợt sale lớn hoặc drop giới hạn, cần tối ưu khả năng mở rộng (scalability, caching, CDN).
\end{itemize}
\subsubsection{Zara}
\subsubsubsection{Giới thiệu Zara}
Zara, thương hiệu chủ lực của tập đoàn Inditex (Tây Ban Nha), là minh chứng xuất sắc nhất cho mô hình "thời trang siêu tốc" (Instant Fashion) trên quy mô toàn cầu. Khác biệt với đa số đối thủ, Zara vận hành chuỗi cung ứng theo mô hình liên kết dọc (Vertical Integration), kiểm soát chặt chẽ hầu hết các khâu từ thiết kế, nhuộm vải, sản xuất cho đến phân phối thay vì phụ thuộc hoàn toàn vào gia công thuê ngoài. Hệ thống kho vận của Zara được quản lý tập trung với công nghệ RFID tiên tiến, cho phép luân chuyển hàng hóa chính xác giữa kênh Online và Offline. Điểm mạnh cốt lõi của thương hiệu nằm ở tốc độ vòng quay sản phẩm kỷ lục (Lead time chỉ từ 2-3 tuần) và quy trình ra quyết định dựa trên dữ liệu thực (Data-driven) từ phản hồi khách hàng, giúp giảm thiểu tối đa hàng tồn kho và tối ưu hóa lợi nhuận.\cite{Zara}

\begin{figure}[H]
	\centering
	\includegraphics[width=0.8\textwidth]{Images/Brand_Imgs/Zara.png}
	\vspace{0.5cm}
	\caption{Trang chủ Zara\cite{Zara}}
	\label{fig: Trang chủ Zara}
\end{figure}

\subsubsubsection{Đặc điểm kỹ thuật nổi bật}

Hệ thống web/app Zara được xây dựng nhằm phản ánh triết lý "thời trang nhanh" với các tính năng số hoá nổi bật:

\begin{itemize}
  \item \textbf{Trải nghiệm trực quan \& cập nhật nhanh:} Trang chủ và danh mục luôn ưu tiên mục "THE NEW" (hàng mới về), đồng bộ sản phẩm mới ra mắt toàn cầu; sử dụng hình ảnh/video lookbook chất lượng cao mang phong cách tạp chí.
  \item \textbf{Phân loại sản phẩm chi tiết:} Danh mục không chỉ theo loại trang phục (áo, váy...) mà còn theo chất liệu/kiểu dáng (da, len, ren...), hỗ trợ tìm kiếm xu hướng.
  \item \textbf{Tích hợp Omni-channel:}
    \begin{itemize}
      \item Kiểm tra tồn kho sản phẩm tại cửa hàng gần nhất từ ứng dụng/web.
      \item Đặt hàng online và nhận tại cửa hàng nhanh (Express Pick-up, trong khoảng 120 phút tại một số khu vực).
      \item Hỗ trợ đổi/trả linh hoạt tại cửa hàng hoặc trả hàng tại nhà.
    \end{itemize}
  \item \textbf{Thông tin sản phẩm minh bạch:} Mỗi sản phẩm kèm thông tin thành phần, hướng dẫn chăm sóc, tiêu chuẩn bền vững.
  \item \textbf{Dòng sản phẩm bền vững JOIN LIFE:} Website hỗ trợ gắn nhãn, lọc và giới thiệu các sản phẩm làm từ nguyên liệu hữu cơ/tái chế.
  \item \textbf{Tích hợp chương trình thu gom:} Thông tin về chương trình thu hồi quần áo cũ để tái chế/quyên góp được hiển thị trên web/app.
\end{itemize}

\subsubsubsection{Đánh giá hệ thống}

\textbf{Ưu điểm nổi bật:}
\begin{itemize}
  \item Tốc độ cập nhật sản phẩm online nhanh, đồng bộ với chiến dịch toàn cầu.
  \item Hình ảnh và video lookbook chất lượng cao giúp nâng cao trải nghiệm trực tuyến.
  \item Omni-channel tốt: kiểm tra tồn kho, nhận hàng nhanh tại cửa hàng, đổi trả linh hoạt.
  \item Minh bạch thông tin sản phẩm, gắn nhãn bền vững rõ ràng.
\end{itemize}

\textbf{Nhược điểm \& điểm cần cân nhắc:}
\begin{itemize}
  \item Bộ lọc tìm kiếm còn hạn chế so với các sàn TMĐT lớn (thiếu lọc theo chi tiết như kiểu cổ áo, độ dài).
  \item Không có hệ thống đánh giá/xếp hạng sản phẩm từ khách hàng, khó tham khảo khi chọn size hoặc chất liệu.
  \item Giao diện tập trung quá nhiều vào hình ảnh biên tập (editorial), đôi khi làm giảm tính trực quan khi tìm thông tin sản phẩm cơ bản.
  \item Ứng dụng/website có thể rối rắm trên di động do nhiều hình ảnh dung lượng lớn, gây chậm tải.
  \item Bảng size chưa đồng nhất và chưa có công cụ dự đoán kích cỡ (fit predictor/virtual try-on).
  \item Dịch vụ hỗ trợ online còn hạn chế, chưa tích hợp sâu live chat/hỗ trợ tức thì như các nền tảng TMĐT khác.
\end{itemize}
\subsubsection{Adidas}
\subsubsubsection{Giới thiệu Adidas}
Adidas, tập đoàn đa quốc gia đến từ Đức, là một trong những gã khổng lồ dẫn đầu ngành công nghiệp thời trang thể thao (Sportswear) toàn cầu. Khác với mô hình "thời trang nhanh" của H&M hay Zara, Adidas vận hành dựa trên nền tảng nghiên cứu và phát triển (R&D) chuyên sâu về công nghệ vật liệu và hiệu suất. Trong kỷ nguyên số, Adidas đang quyết liệt thực hiện chiến lược "Own the Game", chuyển dịch trọng tâm từ bán buôn (Wholesale) sang mô hình bán hàng trực tiếp (DTC - Direct to Consumer) thông qua nền tảng thương mại điện tử mạnh mẽ. Hệ thống backend của Adidas phải xử lý một bài toán phức tạp hơn: quản lý vòng đời sản phẩm công nghệ cao, tích hợp dữ liệu từ các ứng dụng rèn luyện sức khỏe (Running app) và cá nhân hóa trải nghiệm mua sắm dựa trên cộng đồng người dùng trung thành (Membership).\cite{Adidas}

\begin{figure}[H]
	\centering
	\includegraphics[width=0.8\textwidth]{Images/Brand_Imgs/Adidas.png}
	\vspace{0.5cm}
	\caption{Trang chủ Adidas\cite{Adidas}}
	\label{fig: Trang chủ Adidas}
\end{figure}

\subsubsubsection{Đặc điểm kỹ thuật nổi bật}

Hệ thống web/app Adidas được thiết kế theo hướng cá nhân hoá trải nghiệm và tích hợp đa kênh, với một số tính năng đáng chú ý:

\begin{itemize}
  \item \textbf{Cá nhân hoá dựa trên AI/ML:} Website và app gợi ý sản phẩm theo lịch sử duyệt web, hành vi mua sắm và sở thích cá nhân.
  \item \textbf{Ứng dụng Adidas App \& Omni-channel:} Người dùng có thể đồng bộ tài khoản, quản lý đơn hàng, theo dõi ưu đãi và tham gia cộng đồng thành viên (AdiClub) cả trên web và app.
  \item \textbf{Thử giày ảo (AR):} Ở một số thị trường, ứng dụng cho phép thử giày trực tuyến bằng camera điện thoại, hỗ trợ quyết định mua hàng online.
  \item \textbf{Tính năng tuỳ chỉnh sản phẩm (miAdidas/Customize):} Cho phép người dùng tự thiết kế giày (chọn màu sắc, chất liệu, chi tiết cá nhân hoá) và đặt hàng trực tuyến.
  \item \textbf{Thiết kế giao diện mobile-first:} Website và ứng dụng tối ưu cho thiết bị di động, hình ảnh sắc nét, điều hướng nhanh, tập trung trải nghiệm UX mượt mà.
  \item \textbf{Chính sách giao hàng và hoàn trả minh bạch:} Thông tin về phí ship, đổi trả miễn phí/thu phí hiển thị rõ ràng, đồng bộ giữa các kênh.
\end{itemize}

\subsubsubsection{Đánh giá hệ thống}

\textbf{Ưu điểm nổi bật:}
\begin{itemize}
  \item Tích hợp công nghệ cá nhân hoá thông minh (AI/ML) nâng cao trải nghiệm.
  \item Ứng dụng AR thử giày online giúp cải thiện quyết định mua hàng.
  \item Hệ thống tuỳ chỉnh sản phẩm (miAdidas) mang lại trải nghiệm cá nhân hoá độc đáo.
  \item Thiết kế mobile-first hiện đại, điều hướng nhanh, tối ưu cho thương mại di động.
  \item Omni-channel mạnh: đồng bộ dữ liệu khách hàng giữa web, app và cửa hàng.
  \item Hệ thống thành viên (AdiClub) tăng tính gắn kết và trung thành với thương hiệu.
\end{itemize}

\textbf{Nhược điểm \& điểm cần cân nhắc:}
\begin{itemize}
  \item Giao diện/UX chưa thật sự nhất quán, đôi khi khó tìm thông tin chi tiết sản phẩm.
  \item Hệ thống thử giày AR mới triển khai ở một số thị trường, chưa phổ biến toàn cầu.
  \item Thiếu công cụ dự đoán kích cỡ nâng cao (Fit Predictor), gây khó khăn cho người mua online do size không đồng nhất giữa các dòng.
  \item Tốc độ tải trang có thể chậm tại các khu vực hạ tầng mạng yếu, do sử dụng nhiều hình ảnh/AR.
  \item Chính sách đổi trả và giao hàng dù minh bạch nhưng còn khác biệt theo từng quốc gia, gây trải nghiệm không đồng nhất.
  \item Quá trình cá nhân hoá đôi khi bị giới hạn do phụ thuộc vào dữ liệu hành vi, chưa đủ linh hoạt nếu khách hàng mới.
\end{itemize}
\subsubsection{Amazon Fashion}
\subsubsubsection{Giới thiệu Amazon Fashion}
Amazon Fashion, nhánh bán lẻ thời trang của tập đoàn công nghệ Amazon, đại diện cho mô hình kinh doanh nền tảng lai (Hybrid Platform) phức tạp bậc nhất thế giới. Khác với các chuỗi bán lẻ truyền thống, Amazon Fashion vừa vận hành các nhãn hiệu riêng (Private Labels) vừa đóng vai trò là sàn giao dịch (Marketplace) cho hàng ngàn thương hiệu thứ ba.\cite{Amazon Fashion}

\begin{figure}[H]
	\centering
	\includegraphics[width=0.8\textwidth]{Images/Brand_Imgs/Amazon-fashion.png}
	\vspace{0.5cm}
	\caption{Trang chủ Amazon Fashion\cite{Amazon Fashion}}
	\label{fig: Trang chủ Amazon Fashion}
\end{figure}

\subsubsubsection{Đặc điểm kỹ thuật nổi bật}

Amazon Fashion là phân nhánh thương mại điện tử thời trang trong hệ sinh thái Amazon, được xây dựng trên nền tảng hạ tầng Amazon toàn cầu. Hệ thống nổi bật nhờ tích hợp logistics mạnh, cá nhân hoá trải nghiệm mua sắm và mạng lưới đánh giá từ cộng đồng:

\begin{itemize}
  \item \textbf{Cá nhân hoá thông minh:} Sử dụng thuật toán học máy để gợi ý sản phẩm dựa trên lịch sử tìm kiếm, duyệt web và hành vi mua hàng.
  \item \textbf{Tích hợp hệ sinh thái Amazon:} Kết nối chặt chẽ với các dịch vụ như Prime (giao hàng nhanh), Fulfillment by Amazon (FBA) và hệ thống kho toàn cầu.
  \item \textbf{Đánh giá \& review phong phú:} Hệ thống feedback từ người mua được tích hợp trực tiếp trong mỗi sản phẩm, giúp nâng cao độ tin cậy.
  \item \textbf{Chính sách đổi trả minh bạch:} Hỗ trợ đổi trả đơn giản, thường miễn phí trong nhiều trường hợp, quy trình được tự động hóa trên hệ thống.
  \item \textbf{Khả năng mở rộng quốc tế:} Hệ thống được xây dựng trên hạ tầng Amazon, cho phép mở rộng sản phẩm và dịch vụ ra nhiều quốc gia dễ dàng.
  \item \textbf{Trải nghiệm mobile-first:} Giao diện tối ưu cho di động, kết hợp công cụ tìm kiếm nâng cao và filter chi tiết.
\end{itemize}

\subsubsubsection{Đánh giá hệ thống}

\textbf{Ưu điểm nổi bật:}
\begin{itemize}
  \item Thuật toán gợi ý sản phẩm hiệu quả, cá nhân hoá mạnh nhờ AI/ML.
  \item Tích hợp sâu với Prime và FBA, đảm bảo tốc độ giao hàng và quản lý logistics vượt trội.
  \item Hệ thống đánh giá, nhận xét lớn giúp người mua dễ tham khảo trước khi quyết định.
  \item Quy trình đổi trả tự động, nhanh gọn, tạo sự tin tưởng cao cho khách hàng online.
  \item Nền tảng Amazon toàn cầu cho phép mở rộng thị trường quốc tế thuận lợi.
\end{itemize}

\textbf{Nhược điểm \& điểm cần cân nhắc:}
\begin{itemize}
  \item Cạnh tranh gay gắt, nhiều sản phẩm trùng lặp khiến trải nghiệm tìm kiếm bị loãng.
  \item Vấn đề chất lượng và hàng giả khó kiểm soát, cùng với tình trạng review ảo.
  \item Dịch vụ \textit{“Try Before You Buy”} đã ngừng triển khai từ 31/01/2025, làm giảm lợi thế thử trước khi mua trong ngành thời trang.
  \item Vẫn tồn tại hạn chế về phân loại size/fit, tỉ lệ trả hàng cao do chọn size không chuẩn.
  \item Người bán ít quyền kiểm soát về cách hiển thị sản phẩm, giá và thương hiệu, do phụ thuộc chính sách của Amazon.
\end{itemize}
\subsubsection{Coolmate}
\subsubsubsection{Giới thiệu Coolmate}
Coolmate vận hành theo mô hình thương mại điện tử D2C (Direct-to-Customer) trực tuyến thuần túy, nơi toàn bộ quy trình bán hàng, quản lý sản phẩm, và chăm sóc khách hàng được thực hiện qua nền tảng web. Hệ thống được thiết kế hiện đại, tận dụng công nghệ web và dữ liệu để tối ưu trải nghiệm người dùng.\cite{Coolmate}

\begin{figure}[H]
	\centering
	\includegraphics[width=0.8\textwidth]{Images/Brand_Imgs/Coolmate.png}
	\vspace{0.5cm}
	\caption{Trang chủ Coolmate\cite{Coolmate}}
	\label{fig: Trang chủ Coolmate}
\end{figure}

\subsubsubsection{Đặc điểm kỹ thuật nổi bật}

\begin{itemize}
  \item \textbf{Kiến trúc Web động và tối ưu hiệu năng:} Website được xây dựng bằng các framework hiện đại (ReactJS, NextJS hoặc tương đương), cho phép tải nhanh, tối ưu SEO và hiển thị linh hoạt trên mọi thiết bị.
  \item \textbf{Hệ thống quản lý sản phẩm (PMS):} Toàn bộ thông tin sản phẩm được lưu trữ trong cơ sở dữ liệu có cấu trúc, dễ dàng truy vấn, tìm kiếm và cập nhật. Hệ thống hỗ trợ hiển thị thông tin chi tiết như chất liệu, công nghệ vải và kích thước.
  \item \textbf{Tự động hóa quy trình xử lý đơn hàng:} Các bước từ đặt hàng, thanh toán, đến giao nhận được kết nối qua API với cổng thanh toán (VNPAY, MoMo, ZaloPay) và đơn vị vận chuyển (GHN, GHTK), giúp giảm thời gian xử lý.
  \item \textbf{Gợi ý thông minh dựa trên dữ liệu:} Coolmate triển khai thuật toán đề xuất sản phẩm (recommendation system) và tính năng “Smart Size” giúp người dùng chọn kích cỡ phù hợp dựa trên số đo hoặc lịch sử mua hàng.
  \item \textbf{Tích hợp Chatbot và Live Chat:} Hỗ trợ khách hàng theo thời gian thực qua các nền tảng như Tawk.to hoặc Zendesk, tăng tính tương tác và khả năng phản hồi.
  \item \textbf{Bảo mật và an toàn dữ liệu:} Website sử dụng giao thức HTTPS/TLS, tuân thủ chuẩn PCI DSS cho thanh toán, bảo đảm thông tin cá nhân của người dùng.
  \item \textbf{Phân tích và theo dõi hiệu suất:} Hệ thống tích hợp Google Analytics hoặc công cụ tương tự để theo dõi hành vi người dùng, tỉ lệ chuyển đổi và hiệu quả chiến dịch quảng cáo.
\end{itemize}

\subsubsubsection{Đánh giá hệ thống}

\textbf{Ưu điểm nổi bật}

\begin{itemize}
  \item \textbf{Kiến trúc dễ mở rộng (Scalable Architecture):} Hệ thống web thuần túy cho phép tích hợp nhanh các tính năng mới như AI gợi ý sản phẩm, phân tích dữ liệu hoặc chương trình khách hàng thân thiết.
  \item \textbf{Hiệu năng xử lý cao:} Các quy trình tự động giúp rút ngắn thời gian đặt hàng và giao hàng, nâng cao trải nghiệm người dùng.
  \item \textbf{Tối ưu giao diện người dùng (UI/UX):} Giao diện đơn giản, trực quan, tập trung vào hành động mua sắm, giúp giảm tỷ lệ thoát trang.
  \item \textbf{Dữ liệu tập trung và dễ phân tích:} Hệ thống CRM và kho dữ liệu cho phép theo dõi hành vi người dùng, phục vụ thuật toán gợi ý và dự báo nhu cầu.
  \item \textbf{Khả năng tích hợp cao:} Dễ kết nối với API thanh toán, hệ thống vận chuyển, và nền tảng tiếp thị tự động (marketing automation).
\end{itemize}

\textbf{Nhược điểm \& điểm cần cân nhắc}

\begin{itemize}
  \item \textbf{Phụ thuộc vào hạ tầng Digital:} Mọi lưu lượng truy cập phụ thuộc vào hiệu suất website và quảng cáo trực tuyến, đòi hỏi tối ưu liên tục về tốc độ và SEO.
  \item \textbf{Chi phí hạ tầng tăng theo quy mô:} Khi lưu lượng và dữ liệu người dùng tăng, chi phí vận hành server, CDN và lưu trữ cũng tăng tương ứng.
  \item \textbf{Rủi ro bảo mật dữ liệu:} Việc thu thập và xử lý dữ liệu người dùng yêu cầu tuân thủ nghiêm ngặt các quy định về bảo mật (PDPD, GDPR).
  \item \textbf{Chưa hoàn thiện trải nghiệm đa nền tảng (Omnichannel):} Hệ thống tập trung chủ yếu vào website; ứng dụng di động và đồng bộ giữa các kênh vẫn còn hạn chế.
\end{itemize}
\subsubsection{Hệ thống nhóm đang phát triển - BK Fashion}
\subsubsubsection{Giới thiệu BK Fashion}
BK Fashion là một website bán hàng thời trang trực tuyến hướng đến giới trẻ,
tập trung vào việc mang lại trải nghiệm mua sắm hiện đại, tiện lợi và thông minh.
Website kết hợp công nghệ hỗ trợ gợi ý kích cỡ (Fit Assistant) cùng giao diện thân thiện,
giúp khách hàng dễ dàng lựa chọn trang phục phù hợp phong cách và vóc dáng.

\subsubsubsection{Tính năng nổi bật của BK Fashion}
\begin{itemize}
  \item Fit Assistant + Size Passport: gợi ý kích cỡ chính xác dựa trên thông số cơ thể và lịch sử mua hàng.
  \item Mix \& Match Outfit: gợi ý set trang phục theo phong cách, hoàn cảnh sử dụng.
  \item Buy full set: hỗ trợ hiển thị các sản phẩm xuất hiện trong ảnh mẫu.
  \item Realtime chat: cho phép người dùng và nhân viên chat trực tiếp ngay trên nền tảng
\end{itemize}

\section{Công nghệ sử dụng}

\subsection{Ngôn ngữ lập trình Typescript}

\begin{figure}[H]
	\centering
	\includegraphics[width=0.5\textwidth]{Images/Tech_logo/typescript-logo.png}
	\vspace{0.5cm}
	\caption{Logo của Typescript\cite{Typescript}}
	\label{fig: Logo của Typescript}
\end{figure}

TypeScript là một ngôn ngữ lập trình mã nguồn mở được phát triển bởi Microsoft, đóng vai trò là một siêu tập hợp (superset) của JavaScript. Điểm mạnh cốt lõi của TypeScript nằm ở cơ chế định kiểu tĩnh (Static Typing), cho phép lập trình viên khai báo rõ ràng các kiểu dữ liệu ngay trong quá trình phát triển, giúp phát hiện và ngăn chặn các lỗi logic tiềm ẩn trước khi chương trình được thực thi (Compile-time error checking). \\

Trong dự án xây dựng hệ thống E-commerce này, TypeScript được lựa chọn làm ngôn ngữ chủ đạo để phát triển Backend kết hợp với Framework NestJS. Sự kết hợp này mang lại hiệu quả tối ưu nhờ khả năng tương thích tuyệt đối, bởi chính NestJS cũng được xây dựng hoàn toàn trên nền tảng TypeScript. Cụ thể, TypeScript cho phép nhóm phát triển tận dụng triệt để các tính năng nâng cao của Hướng đối tượng (OOP) như Interfaces, Generics và Decorators để thiết kế kiến trúc hệ thống chặt chẽ, dễ bảo trì. Đặc biệt, việc sử dụng TypeScript giúp định nghĩa tường minh các đối tượng dữ liệu (DTO - Data Transfer Objects), đảm bảo tính nhất quán của dữ liệu đầu vào/đầu ra giữa Client và Server, đồng thời hỗ trợ tối đa các công cụ nhắc lệnh thông minh (IntelliSense) giúp tăng tốc độ viết mã và giảm thiểu sai sót kỹ thuật. \\ 

\subsection{Nestjs Framework}

\begin{figure}[H]
	\centering
	\includegraphics[width=0.5\textwidth]{Images/Tech_logo/nestjs-logo.png}
	\vspace{0.5cm}
	\caption{Logo của Nestjs\cite{Nestjs}}
	\label{fig: Logo của Nestjs}
\end{figure}

NestJS là một framework mã nguồn mở hiện đại chạy trên nền tảng Node.js, được thiết kế để xây dựng các ứng dụng phía máy chủ (server-side) với yêu cầu cao về hiệu năng và khả năng mở rộng (Scalability). Điểm khác biệt cốt lõi của NestJS so với các framework Node.js truyền thống (như Express) nằm ở việc nó cung cấp một kiến trúc module hóa (Modular Architecture) chặt chẽ ngay từ đầu. Kiến trúc này, được lấy cảm hứng từ Angular, khuyến khích lập trình viên tổ chức mã nguồn thành các thành phần độc lập như Modules, Controllers và Services, giúp giải quyết triệt để vấn đề mã nguồn lộn xộn ("Spaghetti code") khi dự án phát triển lớn. \\

Bên cạnh đó, NestJS tận dụng tối đa sức mạnh của TypeScript và áp dụng các nguyên lý thiết kế phần mềm hiện đại như Dependency Injection (DI) và Inversion of Control (IoC). Cơ chế này không chỉ giúp quản lý các sự phụ thuộc giữa các thành phần một cách lỏng lẻo (loose coupling) mà còn tạo điều kiện thuận lợi cho việc kiểm thử đơn vị (Unit Testing) và bảo trì hệ thống. Với hệ sinh thái phong phú hỗ trợ sẵn (out-of-the-box) cho việc tích hợp ORM (như Prisma), xác thực (Authentication) và tài liệu hóa API (Swagger), NestJS là giải pháp nền tảng vững chắc để phát triển backend cho hệ thống E-commerce phức tạp. \\

\subsection{Prisma ORM}

\begin{figure}[H]
	\centering
	\includegraphics[width=0.3\textwidth]{Images/Tech_logo/prisma-logo.png}
	\vspace{0.5cm}
	\caption{Logo của Prisma\cite{Prisma}}
	\label{fig: Logo của Prisma}
\end{figure}
Prisma là một công cụ ánh xạ quan hệ đối tượng (ORM - Object-Relational Mapping) thế hệ mới dành cho Node.js và TypeScript, được lựa chọn làm lớp trung gian để tương tác với cơ sở dữ liệu. Khác biệt với các ORM truyền thống dựa trên Class/Decorator (như TypeORM hay Sequelize), Prisma tiếp cận bài toán theo hướng "Schema-First". Cụ thể, hệ thống dữ liệu được định nghĩa minh bạch qua tệp tin schema.prisma, từ đó Prisma tự động sinh ra Prisma Client - một trình xây dựng truy vấn (Query Builder) mạnh mẽ và an toàn kiểu (Type-safe) tuyệt đối. \\

Trong khuôn khổ dự án này, việc tích hợp Prisma mang lại lợi thế kép: Thứ nhất, nó đảm bảo tính toàn vẹn dữ liệu ở mức mã nguồn, giúp phát hiện lỗi sai lệch tên bảng hay kiểu dữ liệu ngay trong quá trình biên dịch (Compile-time) thay vì chờ đến khi chạy (Runtime). Thứ hai, tính năng Prisma Migrate giúp quản lý và đồng bộ hóa các thay đổi trong cấu trúc cơ sở dữ liệu PostgreSQL một cách hệ thống, giúp quy trình phát triển và triển khai (Deployment) trở nên trơn tru và ít rủi ro hơn. \\


\subsection{Hệ quản trị cơ sở dữ liệu PostgreSQL}

\begin{figure}[H]
	\centering
	\includegraphics[width=0.3\textwidth]{Images/Tech_logo/PostgreSQL_logo.png}
	\vspace{0.5cm}
	\caption{Logo của PostgreSQL\cite{PostgreSQL}}
	\label{fig: Logo của PostgreSQL}
\end{figure}
PostgreSQL được lựa chọn làm hệ quản trị cơ sở dữ liệu (RDBMS) chính cho dự án nhờ vị thế là một trong những nền tảng mã nguồn mở mạnh mẽ và tiên tiến nhất hiện nay. Trong bối cảnh đặc thù của hệ thống thương mại điện tử, nơi dữ liệu về đơn hàng, kho vận và thanh toán yêu cầu độ chính xác tuyệt đối, PostgreSQL thể hiện ưu thế vượt trội nhờ khả năng tuân thủ nghiêm ngặt chuẩn ACID (Atomicity, Consistency, Isolation, Durability). Cơ chế này đảm bảo tính toàn vẹn của dữ liệu trong mọi tình huống, ngăn chặn triệt để các rủi ro về mất mát hoặc sai lệch thông tin giao dịch tài chính ngay cả khi hệ thống gặp sự cố bất ngờ. \\

Về mặt hiệu năng, PostgreSQL cung cấp khả năng xử lý tối ưu cho các truy vấn phức tạp (Complex Queries) và hỗ trợ đa dạng các chiến lược đánh chỉ mục (Indexing) tiên tiến (như B-Tree, GIN, GiST). Điều này cho phép hệ thống duy trì tốc độ truy xuất nhanh chóng ngay cả khi dữ liệu sản phẩm và lịch sử giao dịch tăng trưởng lớn theo thời gian. \\


\subsection{Nextjs Framework}

\begin{figure}[H]
	\centering
	\includegraphics[width=0.3\textwidth]{Images/Tech_logo/next-js-logo.png}
	\vspace{0.5cm}
	\caption{Logo của Nextjs\cite{Nextjs}}
	\label{fig: Logo của Nextjs}
\end{figure}
Đối với việc phát triển giao diện người dùng (Frontend), nhóm thực hiện quyết định sử dụng Next.js, một framework mã nguồn mở được xây dựng trên nền tảng thư viện React. Ưu điểm vượt trội khiến Next.js trở thành lựa chọn hàng đầu cho dự án thương mại điện tử này chính là khả năng hỗ trợ mạnh mẽ cơ chế Server-Side Rendering (SSR). \\

Khác với các ứng dụng Single Page Application (SPA) truyền thống thường chỉ tải nội dung phía người dùng (Client-Side Rendering), gây khó khăn cho các robot tìm kiếm trong việc đọc hiểu dữ liệu, Next.js cho phép khởi tạo và trả về nội dung HTML hoàn chỉnh ngay từ phía máy chủ. Cơ chế này đóng vai trò then chốt trong việc tối ưu hóa công cụ tìm kiếm (SEO), giúp Google và các công cụ khác dễ dàng thu thập dữ liệu (crawl) và lập chỉ mục (index) cho hàng ngàn trang chi tiết sản phẩm. Nhờ đó, thứ hạng hiển thị của website được cải thiện, gia tăng khả năng tiếp cận khách hàng tiềm năng tự nhiên. Đồng thời, SSR cũng giúp giảm thiểu thời gian tải trang ban đầu (First Contentful Paint), mang lại trải nghiệm mượt mà ngay lập tức cho người dùng. \\

\subsection{Shadcn / ui}

\begin{figure}[H]
	\centering
	\includegraphics[width=0.3\textwidth]{Images/Tech_logo/shadcn-ui-logo.png}
	\vspace{0.5cm}
	\caption{Logo của Shadcn / ui\cite{Shadcn / ui}}
	\label{fig: Logo của Shadcn / ui}
\end{figure}
Để xây dựng hệ thống giao diện người dùng (UI) hiện đại và nhất quán, nhóm phát triển đã lựa chọn shadcn/ui. Đây không phải là một thư viện thành phần (Component Library) truyền thống (như Material UI hay Ant Design) mà được định nghĩa là một tập hợp các thành phần tái sử dụng (collection of re-usable components). \\

Về mặt kỹ thuật, shadcn/ui được xây dựng dựa trên sự kết hợp giữa Radix UI (cung cấp các thành phần headless đảm bảo tính truy cập - Accessibility và các chức năng nền tảng) và Tailwind CSS (dùng để định kiểu giao diện). Điểm ưu việt nhất khiến shadcn/ui phù hợp với dự án này là cơ chế "Sao chép và Dán" (Copy and Paste) thay vì cài đặt qua npm như một gói phụ thuộc đóng kín. Điều này trao cho lập trình viên quyền kiểm soát tuyệt đối đối với mã nguồn của từng thành phần (Button, Input, Dialog...), cho phép tùy chỉnh sâu về logic và giao diện để phù hợp nhất với nhận diện thương hiệu của hệ thống mà không gặp trở ngại về việc ghi đè (override) CSS phức tạp. \\


\subsection{Swagger}

\begin{figure}[H]
	\centering
	\includegraphics[width=0.3\textwidth]{Images/Tech_logo/swagger-logo.png}
	\vspace{0.5cm}
	\caption{Logo của Swagger\cite{Swagger}}
	\label{fig: Logo của Swagger}
\end{figure}
Trong kiến trúc hệ thống phân tách giữa Frontend và Backend, việc duy trì sự thống nhất về chuẩn giao tiếp dữ liệu là vô cùng quan trọng. Do đó, nhóm sử dụng Swagger (dựa trên chuẩn OpenAPI) làm giải pháp tài liệu hóa API tự động. \\

Thay vì viết tài liệu thủ công dễ dẫn đến sai sót và lạc hậu so với mã nguồn, Swagger được tích hợp trực tiếp vào NestJS để tự động sinh ra tài liệu mô tả chi tiết các điểm cuối (Endpoints), cấu trúc dữ liệu đầu vào (DTO) và đầu ra. Lợi ích lớn nhất của Swagger là cung cấp giao diện Swagger UI trực quan, đóng vai trò như một cầu nối giữa đội ngũ phát triển Backend và Frontend. Nó cho phép các lập trình viên Frontend đọc hiểu nghiệp vụ, và đặc biệt là thực hiện kiểm thử (Test) các API ngay trên trình duyệt web để xác thực dữ liệu trước khi tiến hành ghép nối vào giao diện ứng dụng. \\


\subsection{Git \& Github}

\begin{figure}[H]
	\centering
	\includegraphics[width=0.3\textwidth]{Images/Tech_logo/git-logo.png}
	\vspace{0.5cm}
	\caption{Logo của Git\cite{Git}}
	\label{fig: Logo của Git}
\end{figure}


\begin{figure}[H]
	\centering
	\includegraphics[width=0.3\textwidth]{Images/Tech_logo/github-logo.png}
	\vspace{0.5cm}
	\caption{Logo của Github\cite{Github}}
	\label{fig: Logo của Github}
\end{figure}

Để đảm bảo tính toàn vẹn của mã nguồn và tối ưu hóa quy trình làm việc nhóm, dự án sử dụng Git làm hệ thống quản lý phiên bản phân tán (Distributed Version Control System), kết hợp với nền tảng lưu trữ GitHub. \\

Git đóng vai trò cốt lõi trong việc theo dõi lịch sử thay đổi (Change logs), cho phép nhóm phát triển tạo ra các nhánh riêng biệt (Branching) để phát triển tính năng mới mà không làm ảnh hưởng đến luồng hoạt động chính (Master/Main branch) của hệ thống. Đồng thời, GitHub được sử dụng như một kho lưu trữ tập trung (Central Repository), hỗ trợ các thao tác hợp nhất mã nguồn (Merge Request), review code giữa các thành viên và giải quyết các xung đột (Merge Conflict) một cách hiệu quả. Đây cũng là tiền đề quan trọng để thiết lập các quy trình tích hợp liên tục (CI/CD) trong tương lai. \\

\subsection{AWS S3}

\begin{figure}[H]
	\centering
	\includegraphics[width=0.3\textwidth]{Images/Tech_logo/amazon-s3-logo.png}
	\vspace{0.5cm}
	\caption{Logo của AWS S3\cite{AWS S3}}
	\label{fig: Logo của AWS S3}
\end{figure}

Để giải quyết bài toán quản lý dữ liệu phi cấu trúc (Unstructured Data), hệ thống tích hợp Amazon S3 (Simple Storage Service) – dịch vụ lưu trữ đối tượng hàng đầu của nền tảng điện toán đám mây AWS. Thay vì phương pháp truyền thống là lưu trữ tệp tin media (hình ảnh sản phẩm, video, ảnh đại diện) trực tiếp trên ổ cứng của máy chủ ứng dụng (Web Server), việc sử dụng S3 mang lại giải pháp lưu trữ tách biệt và chuyên dụng. \\

S3 được thiết kế với độ bền dữ liệu lên tới 99,999999999\% (11 số 9) và khả năng mở rộng vô hạn, cho phép hệ thống lưu trữ hàng triệu hình ảnh sản phẩm mà không ảnh hưởng đến hiệu năng của Backend. Hơn nữa, việc tách rời lớp lưu trữ này giúp hệ thống dễ dàng triển khai trên nhiều môi trường khác nhau (Container/Docker) mà không lo ngại vấn đề đồng bộ dữ liệu tĩnh (Static Assets), đồng thời tối ưu hóa băng thông tải trang nhờ khả năng phục vụ nội dung tốc độ cao. \\


\subsection{VNPAY}

\begin{figure}[H]
	\centering
	\includegraphics[width=0.3\textwidth]{Images/Tech_logo/vnpay-logo.jpg}
	\vspace{0.5cm}
	\caption{Logo của VNPAY\cite{VNPAY}}
	\label{fig: Logo của VNPAY}
\end{figure}

Đối với phân hệ xử lý giao dịch trực tuyến, hệ thống đã tích hợp Cổng thanh toán VNPAY – đơn vị trung gian thanh toán hàng đầu và phổ biến nhất tại Việt Nam hiện nay. VNPAY đóng vai trò là cầu nối an toàn giữa website thương mại điện tử và mạng lưới ngân hàng rộng lớn, cho phép khách hàng hoàn tất đơn hàng nhanh chóng thông qua đa dạng các phương thức như: Quét mã VNPAY-QR trên ứng dụng Mobile Banking, sử dụng thẻ ATM nội địa (Napas) hoặc thẻ thanh toán quốc tế (Visa, Mastercard, JCB). \\

Về mặt kỹ thuật, VNPAY cung cấp bộ API (Application Programming Interface) chuẩn hóa, dễ dàng tích hợp vào hệ thống Backend (NestJS). Quy trình thanh toán được thiết kế theo mô hình chuyển hướng (Redirect), đảm bảo an toàn tuyệt đối cho người dùng vì mọi thông tin nhạy cảm (số thẻ, mã OTP) đều được nhập trực tiếp trên giao diện bảo mật của ngân hàng hoặc cổng thanh toán, hệ thống website bán hàng không lưu trữ các dữ liệu này, giúp tuân thủ các tiêu chuẩn bảo mật khắt khe trong ngành tài chính.\\