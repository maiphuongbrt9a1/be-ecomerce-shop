\section*{Tóm tắt}
	\thispagestyle{empty}

Đề tài "Phát triển hệ thống bán hàng trực tuyến cho cửa hàng thời trang" được thực hiện bởi các sinh viên Võ Mai Phương, Trương Tấn Sang. Đồ án được sự giám sát và tận tình chỉ bảo của các thầy Trương Tuấn Anh và Nguyễn Minh Tâm. \\

Hiện tại nhóm đã và đang nghiên cứu để triển khai một hệ thống bán hàng trực tuyến dưới dạng ứng dụng Web (Web Application).Hệ thống được phát triển dành cho các cửa hàng thời trang có quy mô dưới 10 nhân viên và chỉ có 2 cơ sở hoạt động kinh doanh.\\

Sản phẩm thời trang mà cửa hàng đang buôn bán là quần áo, giày dép, mũ, nón các loại dành cho cả nam và nữ. Ngoài ra còn có thêm các loại phụ kiện thời trang khác cho nam và nữ như thắt lưng, ví, túi, balo. Tất cả các sản phẩm của shop đều phục vụ cho cả nam và nữ trong độ tuổi từ 15 tuổi đến dưới 30 tuổi. \\ 

Đến với đề tài "Phát triển hệ thống bán hàng trực tuyến cho cửa hàng thời trang", các người dùng chính của hệ thống đó là người khách hàng (customer), người nhân viên cửa hàng (staff) và người quản trị viên cửa hàng (admin). Đối với mỗi vai trò, thì họ có những nhu cầu khác nhau và sẽ có các chức năng đặc thù của hệ thống phục vụ họ.\\ 

Thứ nhất là vai trò người khách hàng của cửa hàng (customer), họ có thể duyệt qua các danh mục sản phẩm và sản phẩm của cửa hàng. Họ có thể tiến hành tìm kiếm sản phẩm theo từ khóa hoặc bộ lọc được cung cấp. Bên cạnh đó, hệ thống còn hỗ trợ chức năng so sánh thông tin các sản phẩm cùng loại danh mục giúp người khách hàng dễ dàng chọn hàng mà không cần phải phân vân giữa các sản phẩm. Ngoài ra, hệ thống sẽ hỗ trợ đề xuất các sản phẩm hot trend, sản phẩm phù hợp với sở thích người dùng, hỗ trợ gợi ý mua full set đồ, hỗ trợ gợi ý kích cỡ phù hợp. Không những vậy, người khách hàng có thể quản lý được giỏ hàng của mình, có thể mua hàng nhanh ngay trên trang thông tin chi tiết sản phẩm, hoặc có thể mua hàng từ giỏ hàng hoặc cũng có thể mua hàng từ các đơn hàng đã mua trước đó. Việc mua hàng cũng sẽ được hỗ trợ một cách thuận tiện nhất bằng việc tích hợp nhiều phương thức thanh toán (COD, VNPAY, MOMO), tích hợp voucher giảm giá, hỗ trợ việc lưu và dùng lại các địa chỉ giao hàng đã dùng. Tiếp theo đó, trong quá trình mua hàng nếu gặp bất cứ phân vân nào cần được sự hỗ trợ của cửa hàng, người khách hàng có thể dùng tính năng nhắn tin realtime được tích hợp để liên hệ đến cửa hàng mà không tốn chi phí cước viễn thông. Trong các trường hợp khách hàng không hài lòng với đơn hàng được giao hay đã thay đổi nhu cầu mua hàng người khách hàng hoàn toàn có thể hủy đơn hàng hay tạo yêu cầu trả hàng hoàn tiền để bảo vệ quyền lợi của mình. \\

Thứ hai, là vai trò của người quản trị viên cửa hàng (admin), họ có toàn quyền trong việc quản lý danh mục sản phẩm, sản phẩm, voucher, thông tin của người nhân viên cửa hàng. Bên cạnh đó, người quản trị viên có thể quản lý việc xử lý và vận chuyển đơn hàng đến tay người dùng. Tiếp đó, hệ thống còn hỗ trợ người quản trị viên trong việc quản lý hàng tồn kho, trong bất kỳ thời điểm nào cũng có thể kiểm tra được lượng hàng tồn kho của bất kỳ sản phẩm nào trong cửa hàng. Không chỉ có vậy, người quản trị viên có thể quản lý được doanh thu của cửa hàng, qua đó admin có thể xem được  doanh thu hằng ngày, tuần, tháng, năm. Xem các mặt hàng bán chạy trong ngày, tuần, tháng, năm. Xem được thông tin nhân viên nào của cửa hàng là nhân viên bán hàng với doanh số cao nhất. Ngoài ra, khi có bất kỳ yêu cầu hủy đơn - hoàn tiền, trả hàng - hoàn tiền người admin có toàn quyền duyệt, chuyển khoản thanh toán lại cho những người dùng đã tạo các yêu cầu này. Trong quá trình vận hành người quản trị viên có thể dùng tính năng nhắn tin để liên hệ với các nhân viên khác hay tư vấn cho khách hàng. \\

Thứ ba, là vai trò của người nhân viên cửa hàng (staff), họ sẽ có thể quản lý hàng tồn kho, quản lý việc xử lý đóng đơn hàng và vận chuyển đơn hàng, quản lý thông tin cá nhân cửa mình. Đối với chức năng xử lý các yêu cầu hủy đơn - hoàn tiền hoặc trả hàng - hoàn tiền của khách hàng, người nhân viên đóng vai trò là người kiểm duyệt tính hợp lệ của yêu cầu so với chính sách của cửa hàng. Trong trường hợp các yêu cầu này cung cấp đầy đủ bằng chứng về việc mở hàng lần đầu và đơn hàng giao sai yêu cầu người nhân viên sẽ chấp nhận các yêu cầu này và sau đó việc hoàn tiền sẽ hoàn toàn do người quản trị viên thực hiện. Người nhân viên không thực hiện việc chuyển khoản hoàn tiền. Trong trường hợp các yêu cầu không hợp lệ người nhân viên sẽ từ chối các yêu cầu điều này sẽ làm giảm tải khối lượng công việc cho người quản trị viên. Nếu có sai sót trong quá trình vận hành người nhân viên hoàn toàn có thể dùng tính năng nhắn tin để liên hệ với admin và khách hàng để ghi nhận ý kiến và giải quyết. \\

Để thực hiện được hệ thống, nhóm tiến hàng xây dựng hệ thống theo kiến trúc Three tier. Nhóm chọn kiến trúc này là vì kiến trúc này tách biệt giao diện người dùng với quy trình xử lý nghiệp vụ và với lưu trữ dữ liệu. Điều này cho phép nâng cấp và thay thế một tầng bất kỳ mà không làm ảnh hưởng đến các tầng còn lại. Ở phía backend các công nghệ mà nhóm chọn là NestJS, PrismaORM kết nối PostgreSQL, xác thực người dùng bằng JWT và Google OAuth, lưu trữ video và hình ảnh sản phẩm trên AWS S3, và tích hợp thanh toán VNPay. Kiến trúc module hoá ở backend (sản phẩm, đơn hàng, vận chuyển, khuyến mãi, người dùng,truyền thông) giúp tách bạch trách nhiệm, dễ mở rộng và bảo trì. Bộ APIs tuân thủ chuẩn REST và có tài liệu hoá OpenAPI, thuận tiện cho phát triển front-end và kiểm thử. Ở phía front-end nhóm chọn Nextjs, shadcn / ui để có thể triển nhanh chóng được giao diện người dùng song vẫn đảm bảo tính trải nghiệm nhất quán. 
\clearpage
\pagenumbering{arabic}
